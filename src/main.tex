\documentclass{article}
\usepackage[utf8]{inputenc}
\usepackage[brazil]{babel}
\usepackage{setspace}
\usepackage{mathtools}
\usepackage{pgfplots}
\usepackage{listings}
\usepackage{xcolor}
\usepackage{natbib}
\usepackage{graphicx}

\DeclarePairedDelimiter\ceil{\lceil}{\rceil}
\DeclarePairedDelimiter\floor{\lfloor}{\rfloor}

\definecolor{codegreen}{rgb}{0,0.6,0}
\definecolor{codegray}{rgb}{0.5,0.5,0.5}
\definecolor{codepurple}{rgb}{0.58,0,0.82}
\definecolor{backcolour}{rgb}{0.95,0.95,0.92}

\lstdefinestyle{codigo}{
    numberstyle=\tiny,
    basicstyle=\ttfamily\footnotesize,
    breakatwhitespace=false,         
    breaklines=true,                 
    captionpos=b,                    
    keepspaces=true,                 
    numbers=left,                    
    numbersep=5pt,                  
    showspaces=false,                
    showstringspaces=false,
    showtabs=false,                  
    tabsize=2
}

\lstset{style=codigo}

\setstretch{1.5}
\pgfplotsset{width=10cm,compat=1.9}

\pagenumbering{gobble}
\clearpage
\thispagestyle{empty}

\title{Resumo das duas primeiras seções do artigo ``On Understanding Types, Data Abstraction, and Polymorphism"}
\author{Lucas Santiago}
\date{Maio de 2020}



\begin{document}
\maketitle

\begin{abstract}
    A ideia desse artigo é resumir as duas primeiras seções do artigo \emph{On Understanding Types, Data Abstraction, and Polymorphism} de Luca Cardelli e Peter Wegner.

\end{abstract}

\section{Representação dos dados}

\hspace{10pt} A memória por si é apenas um conjunto de valores em binário, sem distinção entre caracteres, valores numéricos, ponteiros ou programas inteiros. A interpretação dos dados é feita de forma externa, por alguma função, por exemplo.
O nome dos tipos são criados ao conseguir unir um conjunto de valores por uma propriedade em comum. Por exemplo, inteiros são números representados sem parte decimal e números flutuantes são valores inteiros mais uma parte decimal.

A ideia de ter-se criado línguas de programação de tipagem forte é proteger a variável de ser interpretada de forma errônea, o uso, por exemplo, de um inteiro com um ponteiro poderia ser desastroso ao se acessar arbitrariamente os dados de outro programa em execução. 
\end{document}